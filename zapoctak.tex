\documentclass[11pt,twocolumn,a4paper]{article}

\usepackage[T1]{fontenc}
\usepackage[utf8]{inputenc}
\usepackage[czech]{babel}
\usepackage{amsfonts}
\usepackage{amssymb}
\usepackage{amsmath}
\usepackage{amsthm}
\usepackage{latexsym}
\usepackage{listings}
\usepackage{multicol}
\usepackage{graphicx} 
\usepackage{listings}


\newcounter{vety}
\newtheorem*{definice}{Definice}
\newtheorem{vetal}[vety]{Věta L}
\newtheorem{vetat}[vety]{Věta T}
\newtheorem{vetabd}[vety]{Věta BD}
\newtheorem*{dusledek}{Důsledek}
\newtheorem*{opakovani}{Opakování}
\newtheorem*{pozorovani}{Pozorování}
\newtheorem*{poznamka}{Poznámka}
\newtheorem*{tvrzeni}{Tvrzení}
\newtheorem*{priklad}{Příklad}
\newtheorem*{dukaz}{Důkaz}


\lstset{language=Haskell}

\begin{document}

\begin{multicols}{2}
\title{Odhad TSP - Haskell \\ Zápočtový program na Neprocedurální programování}
\author{Tomáš Krejčí <tomas789@gmail.com>}
\maketitle
\end{multicols}

\maketitle

\section{Problém obchodního cestujícího}

TSP (Traveling Salesman Problem, česky Problém obchodního cestujícího) je úloha o hledání nejkratší hamiltonovské kružnice v zadaném grahu s ohodnocenými hranami.

Úloha je inspirována problémem, kdy má obchodní cestující objet $N$ měst tak, aby každé navštívil právě jednou a zároveň aby najezdil co nejméně. Převedeno do řeči teorie grafů máme graf $G = (V, E, c)$ takový, že $|V| = N$, $E = {V \choose 2}$ a $c\ :\ E^2 \to \mathbb{R}$. Případně $c$ může zobrazovat do libovolné jiné množiny na níž máme definovány operace $\oplus$ a $\preceq$.

\subsection{Speciální případy}

Speciálním případem TSP může být například situace, kdy $e$ splňuje trojúhelníkovou nerovnost
$$\forall x,y,z \in V \ : \ c(\{x,z\}) \preceq c(\{x,y\}) \oplus c(\{y,z\})$$
případně $c$ je metrika.

Dalším speciálním případem je situace, kdy $V \subset \mathbb{R}^n$. $c$ poté může být například klasická euklidovská norma.

\subsection{Heuristiky}

TSP je výpočetně velmi složiný problém, kdy k nalezení optimálního řešení obsahujícího $N$ měst je třeba vyšetřit exponenciálně mnoho měst. 

Nalezení nějaké dobré aproximace však může být mnohem snazší úloha. Jednou z prvních možností je \emph{nearest-neighbour algoritmus} který schematicky zapsáno vypadá následovně
\begin{figure}[h]
\begin{enumerate}
\item $P = v \in V$ libovolně
\item $\bold{while} \ P \neq V \ \bold{begin}$
\item označ $v \in V \backslash P$ s nejmenší vzdáleností k $P$
\item $P = P \cup v$
\item $\bold{end}$
\end{enumerate}
\end{figure}


Další možností jsou takzvané konstrukční heuristiky. Jejich schéma ja takové, že začneme s nějakou podcestou (podmnožina cesty v celém grafu) a postupně ji rozšiřujeme vkládáním nějakého vrcholu na vhodné místo dokud nemáme celou cestu. Tyto heuristiky se liší v několika významných aspektech.
\begin{itemize}
\item Jak vybrat počáteční podcestu
\item Jak zvolit vrchol, který budeme vkládat
\item Kam zvolený vrchol do cesty vložit
\end{itemize}

Počáteční podcesta je obvykle nějaké množina tří libovolně volených vrcholů nebo ve speciálním případě \emph{Euklidovského TSP} je dobrou volbou začít napčíklad s \emph{konvexním obalem} všech bodů, protože posloupnost vrcholů z konvexního obalu je vždy obsažena v optimálním řešení.

Zvolený bod je obvykle vkládán na pozici v podcestě kde bude míc co nejmenší příspěvek na jejím prodloužení.

Možností jak vybrat bod, který se bude vkládat do cesty je několik. Uvedu zde dva.
\begin{itemize}
\item \textbf{Nejlevnější vkládání}: Ze všech bodů vybereme ten, který způsobí nejmenší prodloužení cesty. Tento přístup je inspirovaný hladovými algoritmy.
\item \textbf{Nejvzdálenější vložení}: Vložíme bod, jehož nejmenší vzdálenost od podcesty je největší. Tímto přístupem se snažíme v již pro malé cesty co do počtu vrcholů přiblížit celkovému tvaru cesty.
\end{itemize}

\section{Zadání}

Cílem práce je implementovat nějakou vhodnou heuristiku TSP v programovacím jazyce Haskell.

\section{Implementace}

Program implementuje konstruktivní heuristiku nazvanou \emph{farthest insert} a jako počáteční množinu vrcholů vybírá koncexní obal množiny všech vrcholů. Vrcholy jsou implementovány jako body v $\mathbb{R}^2$ a jejich vzdálenost určuje uživatelem definovaná funkce.

Uveďme nějaké základní typové konstruktory
\begin{verbatim}
data Point a = P a a deriving (Show,Eq)
data Tour a = T [Point a] deriving (Show)
distance :: Floating a => Point a -> Point a -> a
\end{verbatim} 
kde distance je funkce určující vzdálenosti mezi dvěma body.

\section{Použití}

Výsledný program přijímá na vstupu seznam bodů takových, že na každém řádku je jeden bod $P$. Formát řádku je "$P(x)$ $P(y)$". 
Výstup je ve stejném formátu seřazený tak, jak jdou jednotlivé body zasebou ve výsledné cestě.

\end{document}